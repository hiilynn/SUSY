%%% Research Diary - Entry
%%% Template by Mikhail Klassen, April 2013
%%%
\documentclass[11pt,letterpaper]{article}

\usepackage[all]{xy}


\newcommand{\workingDate}{\textsc{2016 $|$ Winter}}
\newcommand{\userName}{Yonsei-HEP-COSMO}
\newcommand{\institution}{Yonsei University}
\usepackage{researchdiary_png}
% To add your univeristy logo to the upper right, simply
% upload a file named "logo.png" using the files menu above.

\begin{document}
% \univlogo


\begin{center}
{\Huge \bfseries SUSY Lectures in 2017 Summer}\\[10mm]
Kwang Sik Jeong \footnotemark[1] \\
\VS
{\SM Noted by Y.J. Park, T.G. Kim} 
\end{center}

\footnotetext[1]{Dept. of Physics, Pusan National University}

    

\tableofcontents

\newpage

\section{Introduction}

\subsection{[L1] Quantum Field Theory}

    : basic framework to study elementary particles and their interactions.

\begin{center}
{\color{blue}{"special relativity + Quantum mechanics"}}
\end{center}

%
\begin{equation*}
 \textnormal{On special relativity} \quad
 \begin{cases}
  \textnormal{spacetime}: (ct,\vec{x}) \\
  \textnormal{force: mediated by fields. (e.g. EM)}
 \end{cases}
\end{equation*}


%\figure
$S$: action \\
$\delta S$ = 0: classical action(least action)

\begin{align*}
 \braket{f|i} \sim \sum_{\text{path}}e^{iS/\hbar} \quad \textnormal{; path integral} \\
 \begin{cases}
  \Delta S \gg \hbar \quad \textnormal{; Classical} \\
  \Delta S \sim \hbar \quad \textnormal{; Quantum}
 \end{cases}
\end{align*}

%

We use the natural unit, $\hbar = c = 1$.
%
\begin{align*}
 \textnormal{Theory} \quad
 \begin{cases}
  \textnormal{action}(S) \\
  \quad \small{\mbox{+}} \\
  \textnormal{regularization} \\
  \quad \small{\mbox{+}} \\
  \textnormal{renormalization}
 \end{cases}
\end{align*}


\subsubsection{$S$(action)}
%
\begin{equation*}
 S = \int \underbrace{d^{4}x}_{dtdxdydz} \mathcal{L}(\phi, \partial_{\mu}\phi)
\end{equation*}
%
where $\phi$ is a field, $\phi(x)$. Lagrange has a symmetry. \\

%

\begin{quote}
i) symmetry \\
- Important role to better understand natural. \\
- Defining an elementary particle according to the behavior of the corresponding field with respect to symmetries. \\
- Determining interactions among particles. \\
(can be hidden; spontaneous breaking) \\
%
\end{quote}

%

{\bf Noether Theorem(classical field theory)} \\
%
: continuous symmetry $\rightarrow$ conserved quantity

\VS

{\bf Symmetry}
%
 \begin{enumerate}
  \item Spacetime Symmetry ($x \rightarrow x'$)
   \begin{enumerate}
    \item Poincar\'{e} Transformation: transformation leaving the line element invariant. \\
    $\rightarrow ds^2 = \eta_{\mu \nu} dx^{\mu}dx^{\nu}$
    \begin{enumerate}
     \item Lorentz transformation; action conservation
     \item Translations; energy-momentum conservation
    \end{enumerate}
    \item General coordinate Transformation \\
    $\quad \rightarrow$ general relativity
   \end{enumerate}
%
  \item Internal Symmetry \\
    $$\Phi^{a}(x) \rightarrow \Phi'^{a}(x) = \tensor{M}{^a _b}\Phi^{b}(x)$$
    where $\Phi^{a}(x)$ is general field and $a,b$ are internal indices.
    \begin{equation*}
     \tensor{M}{^a _b}
     \begin{cases}
      \mbox{global symmetry: $\tensor{M}{^a _b}$ is spacetime-independent} \\
      \mbox{local(gauge) symmetry: $\tensor{M}{^a _b}(x)$ is spacetime-dependent}
     \end{cases}
    \end{equation*}
\end{enumerate}

\VS

QFT: the most general symmetry of the S-matrix(Scattering matrix).

%

\begin{note}
 Coleman-Mandula(1967)
 { \normalfont
  \begin{center}
   Poincar\'{e} symmetry $\bigotimes$ Internal symmetry
  \end{center}
%
where $\bigotimes$ is a direct product.  
 \begin{itemize}
  \item assumption on QFT (local, relativity, 4$D$, \ldots) \\
        \& scattering interacions.
  \item considered only bosonic generator. 
 \end{itemize}
 }
\end{note}

%
Extended to include spinor generators(spin: $\tfrac{1}{2}$, $\tfrac{3}{2}$),
%

\begin{note}
 Haag-Lopusza\'{n}ski-Sohnius(1975)
 { \normalfont
  \begin{center}
   Super-Poincar\'{e} symmetry $\bigotimes$ Internal symmetry
  \end{center}
 }
\end{note}

\VS

\subsubsection{Regularization \& Renormalization}
%
\begin{enumerate}
  \item Regularization
  \begin{enumerate}
    \item UV divergence \\
            : we need to regulate the theory. \\
         (Infinities $\rightarrow$ absorbed by appropriate counter terms.) \\
         e.g. cutoff regulation, dimensional regularization
  \end{enumerate}
  \item Renormalization \\
        : determine how to absorb the infinities into counter terms. \\
        (arbitraty renormalization point $\mu$(unphysical))
   \begin{enumerate}
     \item Perturbation Theory \\
            : expansion w.r.t $\underbrace{\mbox{renormalized parameters.}}_{\SM \mu - \mbox{dep.}}$
             $\rightarrow$ large logs: $\ln(\frac{E}{\mu}), \ln(\frac{M}{\mu})$ \\
             $\rightarrow$ physical amplitude(independent of $\mu$) $\rightarrow$ ``set $\mu \sim E$" \\
     \item RGE(Renormalization Group Equation) \\
            % figure
   \end{enumerate}
\end{enumerate}

\newpage

\subsection{[L2] Effective Field Theory}

[Need Figure!] To make massive particle, we need high energy. So in low energy scale, we can see only small mass particles.
We can't \& don't need to see higher mass particles. 

\VS

{\bfseries $-$ Full Theory (UV)}

$$
\begin{aligned}
	\L_{full} &= \L_{H}(\phi_H, \phi_L) + \L(\phi_L) \\
	&\text{where} \hs 
	\begin{cases}
		\phi_H : \text{Fields describing particles with masses } >  \Lambda \\
		\phi_L : \text{Fields describing particles with masses }  < \Lambda
	\end{cases}
\end{aligned}
$$

\vs

{\bfseries $-$ Effective theory of light fields $\phi_L$}

\quad : Integrating out $\phi_H$ in E.O.M. (Effects of all Heavy particle are in $C_i$)

$$
\begin{aligned}
	\L_{eff} &= \L(\phi_L) + \sum_{k} \frac{C_i}{\Lambda^{k-4}}\O_i^{(k)} (\phi_L) \\
	&\text{where} \hs
	\begin{cases}
		C_i : \text{Wilson Coefficient (dim=0)} \\
		\Lambda : \text{Cut-off sclae of EFT}
	\end{cases}
\end{aligned}
$$

\begin{enumerate}[(1)]
	\item Contribution of $\O_i^{(k)}$ to a process at energy scale $E \gg \Lambda$ \newline
	$$C_i\left( \frac{E}{\Lambda} \right)^{k-4}
	\begin{cases}
		k = 4 : \text{Marginal} \\
		k<4 : \text{Relavent} \\
		k>4 : \text{Irrelavnt - (Almost non-renormalizable)}
	\end{cases}
	$$
	\item Matching (Around $\Lambda$) : Same physical predictions at low energy. \newline
	$$
	\lambda_{eff}(\mu)  = \lambda_{full}(\mu) + (\text{threshold corrections})
	$$ \newline
	e.g. [Need Figure!]
\end{enumerate}

\newpage

\subsection{[L2] Poincar\'{e} Symmetry}

\VS

{\bfseries $-$ Symmetry of the SM}

\vs

$ \quad
\begin{cases}
	\text{
		Poincar\'{e} $
		\begin{cases}
			\text{Lorentz Inv.} \\
			\text{Translation Sym.}		
		\end{cases}
	$
	} \\
	\\
	\text{
		Internal $
		\begin{cases}
			\text{Global: Baryon number, Lepton number, Flavor Sym.} \\
			\text{Gauge/Local: $SU(3)_c\times SU(2)_c\times U(1)_Y$}		
		\end{cases}
	$
	}
\end{cases}
$

\VS\VS

\subsubsection{Group}

\vs

\begin{definition}[Group]
	A set of elements $G=\{g_1, g_2, \cdots\} $ with product operator $*$ s.t
	\begin{enumerate}
		\item Closure: $\forall g_i, g_j \in G,\hs g_i*g_j\in G$
		\item Unit element : $\exists e\in G, \hs e*g = g*e = g \HS \forall g\in G$
		\item Inverse element : $\forall g\in G, \hs \exists g^{-1} \in G \HS s.t \HS g*g^{-1} = g^{-1}*g = e$
		\item Associativity : $\forall g_i, g_j, g_k \in G, \hs g_i*(g_j*g_k) = (g_i * g_j)*g_k$
	\end{enumerate}
\end{definition}

\VS\VS

\subsubsection{Representation}

\VS

\begin{definition}[Representation]
	A map $G \rightarrow R$ where $R = \{D(g)\}$ such that
	\begin{enumerate}
		\item $D(g_1)D(g_2) = D(g_1*g_2)$
		\item $D(e) = 1$
		\item $D(g^{-1}) = D(g)^{-1}$
	\end{enumerate}
	
	$D(g)$ is a linear operator acting on a vector space
	$V=\{v_1, v_2, \cdots \}$.
\end{definition}

\VS\VS

\subsubsection{Poincar\'{e} group}

\vs

{\bfseries $-$ Coordinate trsf leaving $ds^2 = \eta_{\mu\nu}dx^\mu dx^\nu$ is invariant.}

$$
x'=\Lambda x + a \HS
\begin{cases}
	a: \text{Translations} \\
	\Lambda: \text{Lorentz trsf (rotations + boosts)}
\end{cases}
$$
\newpage

{\bfseries$-$ Inifinitesimal transformation : $(\Lambda, a)$}

$$
\begin{aligned}
	&\delta x^\mu = \eps^\mu + \tensor{\omega}{^\mu _\nu}x^\nu \hs (\omega^{\mu\nu} = -\omega^{\nu\mu}\text{; anti-sym}) \\
	\Rightarrow \hs &\delta X^\mu = -i\tensor{\left(\eps_\sigma P^\sigma + \frac{1}{2}\omega_{\rho\sigma}M^{\rho\sigma} \right)}{^\mu _\nu}X^\nu \quad \text{($X$ in Vector space)}
\end{aligned}
$$

In this representations, 
$
\begin{cases}
	P_\mu = i\partial_\mu \\
	M_{\mu\nu} = i(X_\mu\partial_\nu - X_\nu\partial_\mu)
\end{cases}
$
\hs $\rightarrow$ the form of $P$ and $M$ depends on the reps.

\VS\VS

{\bfseries $-$ Poincar\'{e} algebra}

From the composition rule : $(\Lambda_2, a_2)(\Lambda_1, a_1) = (\Lambda_2\Lambda_1, \Lambda_2a_1 + a_2)$, we can derive below algebra.
\begin{enumerate}
	\item $\Commutator[P^\mu][P^\nu] = 0$
	\item $\Commutator[M^{\mu\nu}][P^{\sigma}] = i(P^\mu\eta^{\nu\sigma}-P^\nu\eta^{\mu\sigma})$
	\item $\Commutator[M^{\mu\nu}][M^{\rho\sigma}] = i(M^{\mu\sigma}\eta^{\nu\rho} + M^{\nu\rho}\eta^{\mu\sigma} - M^{\mu\rho}\eta^{\nu\sigma} - M^{\nu\sigma}\eta^{\mu\rho})$
\end{enumerate}

\VS

{\bfseries $-$ Representation on fields : ($x'=\Lambda x + a$)}

\vs

\begin{itemize}
	\item As a vector, $\phi^a(x)\rightarrow \phi'^a(x') = \tensor{R}{^a _b}(\Lambda) \phi^{b} (x)$ \quad ($\phi$ is regarded as scalar for translations.)
	\item As a field(quantum) operator, $\hat{\phi}^a(x)$ \newline
	: Consider transformation of state vector $\ket{\psi}\rightarrow\ket{\psi'}=\hat{U}(\Lambda, a)\ket{\psi}$ \quad ($\hat{U}(\Lambda, a)$ is linear \& unitary.) \newline
	$$\hat{U}(\Lambda,a) = 1-i\eps_\mu \hat{P}^{\mu} -\frac{i}{2}\omega_{\mu\nu}\hat{M}^{\mu\nu}+\cdots$$
	
	\vs
	
	Now consider field operator, we can get below equations.
	
	$$
	\begin{cases}
		\varphi^a(x) = \braket{f|\hat{\phi}^a(x)|g} \HS then \HS 
			\varphi'^a(x) = \braket{f'|\hat{\phi}^a(x)|g'} =\braket{f|\hat{U}^\dagger \hat{\phi}^a(x)\hat{U}|g}\\
			\\
		\varphi'^a(x') = \tensor{R}{^a_b}(\Lambda)\varphi^b(x) = \tensor{R}{^a_b}\braket{f|\hat{\phi}^b(x)|g}
	\end{cases}
	$$ \newline
	$$\therefore \hs \hat{\phi}'^a(x') =  \tensor{R}{^a_b}\hat{\phi}^b(x) = \hat{U}^\dagger \hat{\phi}^a(x')\hat{U}$$
	\begin{center}
		(Caution : It's not unitary equivalence!)
	\end{center}
	
\end{itemize}

\newpage

\section{Supersymmetry}

\subsection{Supersymmetry}

\vs

Symmetry to cancle $\Lambda^2$ - contributions?
$\lambda = y^2?$ \quad {\bfseries [Supersymmetry]}

\VS

{\bfseries $-$ Supersymmetry} : Boson $\leftrightarrow$ Fermion

$$
\begin{aligned}
	\bar{Q}\ket{B} &= \ket{F} \quad (\bar{Q} = Q^\dagger) \\
	Q\ket{F} &= \ket{B} \\
\end{aligned}
$$

* SUSY operator : Anti-commuting spinor (spin 1/2) $\Rightarrow$ Spacetime Symmetry

\vs

\begin{note}[SUSY]
	\normalfont SUSY is the unique extension of Poincar\'{e} algebra within the QM framework. \newline
	(Under reasonable physical assumptions)
\end{note}

\vs

{\bfseries $-$ Supermultiplet : $B, F$ (Superpartner)}
\begin{enumerate}[(i)]
	\item Equal number of fermion and boson D.O.F. \newline
	$$
	n_B(P_\mu) - n_F(P_\mu) \rightarrow \sum_i\braket{i|(-1)^{2s}P^\mu|i}
	$$ \newline
	Since $Q \in (\frac{1}{2}, 0),\, \bar{Q}\in(0,\frac{1}{2})$, then $\ACommutator{Q}{\bar{Q}} \in (\frac{1}{2}, \frac{1}{2}),\, \ACommutator{Q}{\bar{Q}} \sim P^\mu$.
	
	\begin{align*}
	\sum_i\braket{i|(-1)^{2s}P^\mu|i} &= \sum_i\braket{i|(-1)^{2s}QQ^{\dagger}|i} + \sum_i\braket{i|(-1)^{2s}Q^\dagger Q|i} \\
	&= \sum_i\braket{i|(-1)^{2s}QQ^{\dagger}|i} + \sum_{ij}\braket{i|(-1)^{2s}Q^\dagger|j}\braket{j|Q|i} \\
	&= \sum_i\braket{i|(-1)^{2s}QQ^{\dagger}|i} + \sum_{ij}\braket{j|Q|i}\braket{i|(-1)^{2s}Q^\dagger|j} \\
	&= \sum_i\braket{i|(-1)^{2s}QQ^{\dagger}|i} + \sum_j\braket{j|Q(-1)^{2s}Q^\dagger|j} \\
	&= \sum_i\braket{i|(-1)^{2s}QQ^{\dagger}|i} - \sum_j\braket{j|(-1)^{2s}QQ^\dagger|j} = 0
	\end{align*}
	
	$$ \therefore n_B - n_F = 0 \HS for \hs P^\mu \neq 0$$
	
	\item Internal Symmetry \newline
	$$
	\begin{gathered}
	\Commutator[Q][T_i] = 0 \HS \text{where $T_i$ is generator of internal symmetry group} \\
	\Rightarrow\, B\&F \hs  \text{in the supermultiplet : Same internal symmetry charge.} 
	\end{gathered}
	$$
	
	\item Exception : $R$-symmetry \newline
	$$
	\begin{aligned}
		\text{Global}\HS U(1)_R &
		\begin{cases}
			\text{As a vector: } Q\rightarrow e^{i\lambda}Q,\, \bar{Q} \rightarrow e^{-i\lambda}\bar{Q} \\
			\text{As an operator: } \hat{Q}\rightarrow e^{-i\lambda\hat{R}}\hat{Q}e^{i\lambda\hat{R}}
		\end{cases} \\
		& \Rightarrow 
		\begin{cases}
			\Commutator[Q][R] = +Q \\
			\Commutator[\bar{Q}][R] = -Q
		\end{cases}
	\end{aligned}
	$$
\end{enumerate}

\VS\VS

\subsubsection{Super-Poincar\'{e} algebra ($N=1$)}

\vs

{\bfseries $-$ SUSY generators} : $Q_\alpha \sim$ Spinor (spin 1/2)

\HS Lorentz Transform $
\begin{cases}
	\text{As a vector: } Q_\alpha \rightarrow Q'_\alpha = \exp\tensor{\left[-\dfrac{1}{2}\omega_{\mu\nu} \sigma^{\mu\nu} \right]}{_\alpha ^\beta} Q_\beta \\
	\\
	\text{As an operator: } \hat{Q}_\alpha \rightarrow \hat{Q}'_\alpha = \hat{U}^\dagger \hat{Q}_\alpha \hat{U}
\end{cases}
$

\vs
 
For some calculations, we can derive next algebra :
\begin{enumerate}[(i)]
	\item $\Commutator[P^\mu][P^\nu] = 0$
	\item $\Commutator[M_{\mu\nu}][P_\rho] = -i(\eta_{\mu\rho}P_\nu - \eta_{\nu\rho}P_\mu )$
	\item $\Commutator[M_{\mu\nu}][M_{\rho\sigma}] = -i(\eta_{\mu\rho}M_{\nu\sigma}-\eta_{\nu\rho}M_{\mu\sigma} + \eta_{\nu\sigma}M_{\mu\rho} - \eta_{\mu\sigma}M_{\nu\rho})$
	\item $\Commutator[P_\mu][Q_\sigma] = \Commutator[P_\mu][\bar{Q}^{\dot{\alpha}}] = 0$
	\item $\Commutator[M_{\mu\nu}][Q_\alpha] = -i\tensor{(\sigma_{\mu\nu})}{_\alpha^\beta}Q_\beta$
	\item $\Commutator[M_{\mu\nu}][\bar{Q}_{\dot{\beta}}] = -i\tensor{(\bar{\sigma}_{\mu\nu})}{^{\dot{\alpha}}_{\dot{\beta}}}\bar{Q}^{\dot{\beta}}$
	\item $\ACommutator{Q_\alpha}{Q_\beta} = 0$
	\item $\ACommutator{Q_\alpha}{\bar{Q}_{\dot{\beta}}} = 2\tensor{(\sigma^\mu)}{_{\alpha\dot{\beta}}} P_\mu$
\end{enumerate}

Now let's find representations.

\pagebreak

\subsubsection{Superspace Formalism}

{\bfseries $-$ Superspace} ($\chi^\mu, \theta^\alpha, \bar{\theta}_\Dot{\alpha} $) ; Corresponding to ($P^\mu, Q^\alpha, \bar{Q}_\Dot{\alpha}$)

* SUSY : Translation in the Grassmann Coordinates.

$$
\begin{aligned}
	\delta
	\begin{pmatrix}
		\chi^\mu \\ \theta^\alpha \\ \bar{\theta}_\Dot{\alpha}
	\end{pmatrix}
	&\equiv -i(a_\rho P^\rho + \frac{1}{2}\omega_{\rho\sigma}M^{\rho\sigma}-\xi Q - \bar{\xi}\bar{Q})
	\begin{pmatrix}
		\chi^\mu \\ \theta^\alpha \\ \bar{\theta}_\Dot{\alpha}
	\end{pmatrix} \\
	\delta \chi^\mu &= a^\mu + \tensor{\omega}{^\mu_\nu}\chi^\nu - ic(\xi\sigma^\mu\bar{\theta}) + ic^*(\theta\sigma^\mu\bar{\xi}) \\
	\delta\theta^\alpha &= \frac{i}{2}\omega_{\mu\nu}\tensor{(\sigma^{\mu\nu})}{^\alpha_\beta}\theta^\beta + \xi^\alpha \\
	\delta\bar{\theta}_\Dot{\alpha} &= \frac{i}{2}\omega_{\mu\nu} {(\bar{\sigma}^{\mu\nu})_\Dot{\alpha}}^\Dot{\beta}\bar{\theta}_\Dot{\beta} + \bar{\xi}_\Dot{\alpha}
\end{aligned}
$$

\vs

Thus, one finds,
$$
\begin{aligned}
	P_\mu &= i\partial_\mu \\
	Q_\alpha &= -i\PD{}{\theta_\alpha} - c(\sigma^\mu)_{\alpha\Dot{\beta}} \bar{\theta}^\Dot{\beta} \PD{}{x^\mu} \\
	\bar{Q}_\Dot{\alpha} &= i\PD{}{\bar{\theta}^\Dot{\alpha}} + c^* \theta^\beta (\sigma^\mu)_{\beta\Dot{\alpha}} \PD{}{x^\mu}
\end{aligned}
$$

\begin{center}
	where $c$ is determined by the commutation relation.
\end{center}
$$
\begin{aligned}
	&\ACommutator{Q_\alpha}{\bar{Q}_\Dot{\beta}} = 2(\sigma^\mu)_{\alpha\Dot{\beta}} P_\mu \\
	\Rightarrow & Re(c) = 1 \quad \text{(Convenient to set $c=1$)}
\end{aligned}
$$
\end{document}