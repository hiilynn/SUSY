%%% Research Diary - Entry
%%% Template by Mikhail Klassen, April 2013
%%%
\documentclass[11pt,letterpaper]{article}

\usepackage[all]{xy}


\newcommand{\workingDate}{\textsc{2016 $|$ Winter}}
\newcommand{\userName}{Yonsei-HEP-COSMO}
\newcommand{\institution}{Yonsei University}
\usepackage{researchdiary_png}
% To add your univeristy logo to the upper right, simply
% upload a file named "logo.png" using the files menu above.

\begin{document}
% \univlogo


\begin{center}
{\Huge \bfseries SUSY Lectures in 2017 Summer}\\[10mm]
Kwang Sik Jeong \footnotemark[1] \\
\VS
{\SM Noted by Yeji Park, T.G. Kim} 
\end{center}

\footnotetext[1]{Dept. of Physics, Pusan National University}

    

\tableofcontents

\newpage

\section{Introduction}

\subsection{[L1] Quantum Field Theory}

    : basic framework to study elementary particles and their interactions.

\begin{center}
{\color{blue}{"special relativity + Quantum mechanics"}}
\end{center}

%
\begin{equation*}
 \textnormal{On special relativity} \quad
 \begin{cases}
  \textnormal{spacetime}: (ct,\vec{x}) \\
  \textnormal{force: mediated by fields. (e.g. EM)}
 \end{cases}
\end{equation*}


%\figure
$S$: action \\
$\delta S$ = 0: classical action(least action)

\begin{align*}
 \braket{f|i} \sim \sum_{\text{path}}e^{iS/\hbar} \quad \textnormal{; path integral} \\
 \begin{cases}
  \Delta S \gg \hbar \quad \textnormal{; Classical} \\
  \Delta S \sim \hbar \quad \textnormal{; Quantum}
 \end{cases}
\end{align*}

%

We use the natural unit, $\hbar = c = 1$.
%
\begin{align*}
 \textnormal{Theory} \quad
 \begin{cases}
  \textnormal{action}(S) \\
  \quad \small{\mbox{+}} \\
  \textnormal{regularization} \\
  \quad \small{\mbox{+}} \\
  \textnormal{renormalization}
 \end{cases}
\end{align*}


\subsubsection*{\normalfont \bfseries 1-a) $S$(action)}
%
\begin{equation*}
 S = \int \underbrace{d^{4}x}_{dtdxdydz} \mathcal{L}(\phi, \partial_{\mu}\phi)
\end{equation*}
%
where $\phi$ is a field, $\phi(x)$. Lagrange has a symmetry. \\

%

\begin{quote}
i) symmetry \\
- Important role to better understand natural. \\
- Defining an elementary particle according to the behavior of the corresponding field with respect to symmetries. \\
- Determining interactions among particles. \\
(can be hidden; spontaneous breaking) \\
%
\end{quote}

%

{\bf Noether Theorem(classical field theory)} \\
%
: continuous symmetry $\rightarrow$ conserved quantity

\VS

{\bf Symmetry}
%
 \begin{enumerate}
  \item Spacetime Symmetry ($x \rightarrow x'$)
   \begin{enumerate}
    \item Poincar\'{e} Transformation: transformation leaving the line element invariant. \\
    $\rightarrow ds^2 = \eta_{\mu \nu} dx^{\mu}dx^{\nu}$
    \begin{enumerate}
     \item Lorentz transformation; action conservation
     \item Translations; energy-momentum conservation
    \end{enumerate}
    \item General coordinate Transformation \\
    $\quad \rightarrow$ general relativity
   \end{enumerate}
%
  \item Internal Symmetry \\
    $$\Phi^{a}(x) \rightarrow \Phi'^{a}(x) = \tensor{M}{^a _b}\Phi^{b}(x)$$
    where $\Phi^{a}(x)$ is general field and $a,b$ are internal indices.
    \begin{equation*}
     \tensor{M}{^a _b}
     \begin{cases}
      \mbox{global symmetry: $\tensor{M}{^a _b}$ is spacetime-independent} \\
      \mbox{local(gauge) symmetry: $\tensor{M}{^a _b}(x)$ is spacetime-dependent}
     \end{cases}
    \end{equation*}
\end{enumerate}

\VS

QFT: the most general symmetry of the S-matrix(Scattering matrix).

%

\begin{note}
 Coleman-Mandula(1967)
 { \normalfont
  \begin{center}
   Poincar\'{e} symmetry $\bigotimes$ Internal symmetry
  \end{center}
%
where $\bigotimes$ is a direct product.  
 \begin{itemize}
  \item assumption on QFT (local, relativity, 4$D$, \ldots) \\
        \& scattering interacions.
  \item considered only bosonic generator. 
 \end{itemize}
 }
\end{note}

%
Extended to include spinor generators(spin: $\tfrac{1}{2}$, $\tfrac{3}{2}$),
%

\begin{note}
 Haag-Lopusza\'{n}ski-Sohnius(1975)
 { \normalfont
  \begin{center}
   Super-Poincar\'{e} symmetry $\bigotimes$ Internal symmetry
  \end{center}
 }
\end{note}

\VS

\subsubsection*{\normalfont \bfseries 1-b) Regularization \& Renormalization}
%
\begin{enumerate}
  \item Regularization
  \begin{enumerate}
    \item UV divergence \\
            : we need to regulate the theory. \\
         (Infinities $\rightarrow$ absorbed by appropriate counter terms.) \\
         e.g. cutoff regulation, dimensional regularization
  \end{enumerate}
  \item Renormalization \\
        : determine how to absorb the infinities into counter terms. \\
        (arbitraty renormalization point $\mu$(unphysical))
   \begin{enumerate}
     \item Perturbation Theory \\
            : expansion w.r.t $\underbrace{\mbox{renormalized parameters.}}_{\SM \mu - \mbox{dep.}}$
             $\rightarrow$ large logs: $\ln(\frac{E}{\mu}), \ln(\frac{M}{\mu})$ \\
             $\rightarrow$ physical amplitude(independent of $\mu$) $\rightarrow$ ``set $\mu \sim E$" \\
     \item RGE(Renormalization Group Equation) \\
            % figure
   \end{enumerate}
\end{enumerate}

\newpage

\subsection{[L2] Effective Field Theory}



\end{document}